\makeatletter
\renewcommand{\@chapapp}{}% Not necessary...
\newenvironment{chapquote}[2][2em]
  {\setlength{\@tempdima}{#1}%
   \def\chapquote@author{#2}%
   \parshape 1 \@tempdima \dimexpr\textwidth-2\@tempdima\relax%
   \itshape}
  {\par\normalfont\hfill--\ \chapquote@author\hspace*{\@tempdima}\par\bigskip}
\makeatother

\chapter*{Úvod}
\addcontentsline{toc}{chapter}{Úvod} % přidá položku úvod do obsahu

V říjnu 2022 vydal americký youtuber Oats Jenkins video
\href{https://www.youtube.com/watch?v=ePxrVU4M9uA}{I Made BETTER Tic-Tac-Toe}
(Udělal jsem lepší piškvorky) \cite{jenkins22}. V něm popisuje \enquote{americké} 3x3 piškvorky a jejich
problémy. Následně přichází s upravenou verzí hry, ve které jsou dané problémy vyřešeny.
Alternativa se jmenuje TickoaTTwo a například v ní nelze dojít k remíze.
