\chapter{Vývoj implementace}

\section{Použité technologie}

Webová online multiplayer hra je implementovaná pomocí technologií, kterým se
souhrnně říká T3 Stack. Podle oficiální dokumentace \cite{t3stack} úkolem
technologií T3 Stack je

\begin{enumerate}
    \item \B{Řešit problémy:} Neposkytovat všechny technologie, které jsou
        dostupné, ale poskytnout technologie, které jsou základem většiny
        webových aplikací a může být složité je nastavit manuálně.
    \item \B{\enquote{Krvácet} zodpovědně:} Poskytovat technologie, které jsou
        moderní a atraktivní, ale tak, aby v~budoucnu netvořily problémy a
        případně se daly nahradit.
    \item \B{Typová bezpečnost není dobrovolná:} Znalost datových typů pomáhá
        programátorům s~vývojem a bezpečným kódem, proto je v~moderním vývoji
        naprosto nezbytná.
\end{enumerate}


Součástí T3 Stack je nástroj \M{create-t3-app}, který umožňuje vytvořit nový
projekt, ve kterém jsou dané technologie předem nastavené.

Seznam technologií použitých:
\begin{itemize}
    \item \B{TypeScript:} Nadstavba nad programovacím jazykem JavaScript, která
        přidává typové anotace a jejich kontrolu. Tím umožňuje rychlejší a
        bezpečnější vývoj.
    \item \B{Next.js:} Moderní framework pro vytváření serverových částí
        webových  aplikací, slouží jako back-end pro React.
    \item \B{React:} Framework pro vytváření front-endových částí webových
        aplikací.
    \item \B{tRPC:} Knihovna pro bezpečnou a otypovanou komunikaci mezi
        front-endem (React) a back-endem (Next.js).
    \item \B{Prisma:} Knihovna pro komunikaci mezi back-endem (Next.js) a
        databází.
    \item \B{TailwindCSS:} Knihovna pro rychlé stylování webových aplikací.
\end{itemize}

Častou součástí této sady technologií též bývá NextAuth.js, které řeší
přihlašování, registraci a správu uživatelů. Jelikož naše aplikace registraci
uživatelů neumožňuje, není taková knihovna potřeba.

\section{Databáze}

Jako databáze k~aplikaci byla zvolena PostgreSQL, jelikož se v~dnešní době
jedná o~jednu z~nejpoužívanějších a nejoblíbenějších databází \cite{stackoverflow23}.

Schéma databáze je pouze jedna tabulka \M{GameSession} reprezentující jednu hru
mezi dvěma hráči. Vedle ní také enum\footnote{enum v programovacích jazycích a
databázích je množina možných hodnot, kterých může proměnná tohoto typu
nabývat, hodnoty bývají čitelně pojmenovány, interně se však většinou ukládají
jinak než jako text} \M{GridFieldState} reprezentující stav jednoho hracího
políčka (je prázdné, hrál na něj pouze první hráč, hrál na něj pouze druhý
hráč, hráli na něj oba hráči) a enum \M{GameState} reprezentující stav hry
(čeká se na druhého hráče, první hráč je na řadě, druhý hráč je na řadě, první
hráč vyhrál, druhý hráč vyhrál).

Každá \M{GameSession} má identifikátor \M{id}, který je pouze pro interní
použití, uživatelům se však zveřejňuje identifikátor \M{slug} (v~doslovném
překladu slimák, v~informačních technologiích termín používaný pro uživatelsky
čitelný identifikátor), který se používá i v~URL adrese.\label{par:game-id}

\label{par:game-grid}
Hrací plocha je reprezentována jako pole devíti \M{GridFieldState} po řádcích
od levého horního políčka. Políčka jsou v~programu reprezentována čísly 0 až 8.

Hodnoty \M{player1} a \M{player2} ukládají ID hráčů, ve schématu databáze jsou
označeny s otazníkem, protože před připojením daného hráče nabývají hodnoty
\It{NULL} (prázdné hodnoty).

\begin{lstlisting}[language=Prisma,caption={Schéma databáze},label={fig:db-schema}]
enum GridFieldState {
    empty
    player1in
    player2in
    both
}

enum GameState {
    inviting
    player1plays
    player2plays
    player1won
    player2won
}

model GameSession {
    id          String      @id @default(cuid())
    createdAt   DateTime    @default(now())
    updatedAt   DateTime    @updatedAt
    slug        String      @unique
    player1     String?
    player2     String?
    visible     Boolean
    state       GameState   @default(inviting)
    lastPos     Int         @default(-1)
    grid        GridFieldState[]    @default([empty, empty, empty, empty, empty, empty, empty, empty, empty])
}
\end{lstlisting}

\section{Komunikace se serverem}

Komunikace mezi klientem (prohlížečem) a serverem je řešena pomocí tRPC a
WebSocketů. WebSocket umožňuje oboustrannou komunikaci mezi klientem a serverem
bez nutnosti pravidelného dotazování se serveru na změny \cite{mdn-ws}. tRPC
zajišťuje validaci a čtení dat (společně s~knihovnou zod, v~kódu zkracována na
\M{z}) a poskytuje lepší vývojářský prožitek (developer experience) sdílením a
odvozováním typových anotací. Tedy když programátor na back-endu napíše funkci,
která bere jedno číslo a jeden text (string), i v~kódu na front-endu mu
našeptávání v~editoru bude doporučovat číslo a text a ukáže chybu, pokud by
programátor napsal argumenty k~funkci špatně. Tím se odlišuje od tradičních
komunikačních technologiích jako je REST, u~kterých je běžné předat vzdálené
funkci špatné argumenty, obzvláště pokud daný kód píše více vývojářů.

Příklad efektivního využití tRPC je vidět na back-endové funkci, která se stará
o~zahrání tahu hráčem. Funkce přijímá tři argumenty: \M{slug} (specifický
string označující hru, kterou hráč hraje), \M{playerId} (specifický string
označující hráče), \M{buttonIndex} (číslo 0--8 označující herní políčko, do
kterého chce hráč zahrát). Všechny argumenty jsou automaticky přeloženy do
jejich správných datových typů, aby se s~nimi dalo nadále pracovat. Pokud by
přišla nevalidní data (ať už chybou na front-endu nebo například od útočníka),
automaticky se vrátí error.

\begin{lstlisting}[language=JavaScript,caption={Funkce pro zahrání tahu},label={fig:be-make-move}]
makeMove: publicProcedure
    .input(z.object({
        slug: zodSlug(),
        playerId: z.string(),
        buttonIndex: z.number().min(0).max(8),
    }))
    .mutation(async ({ input: { slug, playerId, buttonIndex } }) => {
        console.log(`hrac ${playerId} ve hre ${slug} hraje na policko ${buttonIndex}`);
        // zbytek kodu pro zahrani tahu
\end{lstlisting}

Daná funkce se volá na front-endu po kliknutí na herní políčko se správnými
argumenty. Kdyby byl nějaký argument vynechán, byl předáván jako špatný datový
typ nebo nějaký přebýval, vyhodil by se error.

\begin{lstlisting}[language=JavaScript,caption={Volání funkce pro zahrání tahu},label={lst:make-move}]
await makeMoveMutation.mutateAsync({
    slug,
    playerId,
    buttonIndex,
});
\end{lstlisting}

\subsection{Připojování do hry}

Samotné schéma komunikace v~implementaci TickoaTTwo je rozdělené na několik
částí. Některé jsou rozdílné pro prvního hráče (který hru vytváří a následně se
do ní připojuje) a pro druhého hráče (který se do hry pouze připojuje).

V~případě prvního hráče musí uživatel kliknout na tlačítko \It{CREATE GAME}
(vytvořit hru), kód na server pošle příkaz \M{createNewGame} (vytvořit novou
hru) bez jakýchkoli argumentů, server vygeneruje slug a vytvoří novou hru
s~danou hodnotou \M{slug}, zbylé hodnoty se ponechají výchozí podle schématu
\ref{fig:db-schema}. Informace o~nově vytvořené hře jsou vráceny na klienta a
kód přesměruje prohlížeč na URL \M{/\{newGame.slug\}} (kde \M{\{newGame.slug\}}
je slug nově vytvořené hry, viz \ref{par:game-id}).

\begin{figure}[h]
    \centering
    \includegraphics[width=300px]{img/diagrams/create-new-game.pdf}
    \caption[Schéma komunikace vytváření nové hry]{Schéma komunikace vytváření nové hry\footnotemark}
    \label{fig:create-new-game}
\end{figure}

\footnotetext{všechna schémata komunikace byla vytvořena v programu
diagrams.net \cite{diagrams}}

Oba hráči se do hry připojují stejným způsobem, nějak se dostanou na adresu
\M{/\{slug\}}. Jak je popsáno v~kapitole \ref{sec:user-interface}. V~tu chvíli
kód pošle na server příkaz \M{connectPlayer} (připojit hráče) s~argumentem
\M{slug}, server zjistí, zda hra s~daným slugem existuje a zda je v~ní alespoň
jedno volné místo, a pokud ano, vygeneruje pro hráče unikátní ID, které uloží
do \M{player1} nebo \M{player2} podle toho, zda se do hry připojuje jako první
nebo druhý hráč. Server zpět vrátí informaci o~tom, zda hráč je \M{player1}
nebo \M{player2} a jeho unikátní ID, to front-end následně používá pro
autentifikaci.

V~případě připojení druhého hráče je ještě potřeba nastavit stav hry na
\M{player1plays} a přes WebSocket poslat prvnímu hráči informaci o~změně
herních dat.

\begin{figure}[h]
    \centering
    \includegraphics[width=360px]{img/diagrams/connect-game.pdf}\hspace*{80px}
    \caption[Schéma komunikace připojování do hry]{Schéma komunikace připojování do hry\footnotemark}
    \label{fig:connect-game}
\end{figure}

\footnotetext{čárkovaná šipka nastává pouze v~případě, že se nový hráč
připojuje do už existující hry a tedy je potřeba již připojenému hráči oznámit
změnu stavu hry}

Když oba hráči získají svoje ID, mohou si získat aktuální herní data a připojit
se k~odběru nových herních dat. K~tomu slouží příkazy \M{getGameData} (získat
herní data) a \M{newGameDataSubscription} (nový odběr herních dat). První herní
data pouze jednorázově získá, druhý však přijímá zprávy poslané serverem a když
se herní data změní, vždy přepíše data získaná prvním příkazem daty novými.

Nová data ze strany serveru přicházejí vždy, když jakýkoli hráč zahraje tah
(případně vyhraje) nebo se do hry připojí nový hráč.\footnote{takovou informaci
samozřejmě dostává pouze první hráč, který do hry už připojený je a pro nové
informace už poslouchá, viz obrázek \ref{fig:connect-game}}

\begin{figure}[H]
    \centering
    \includegraphics[width=240px]{img/diagrams/get-game-data.pdf}
    \caption{Schéma komunikace získávání herních dat}
    \label{fig:get-game-data}
\end{figure}

\subsection{Hraní tahů}

V~tuto chvíli jsou oba hráči ve stejné fázi, tedy vidí před sebou hru, mají
herní data (stav hry je \M{player1plays}, protože začíná první hráč) a
poslouchají pro příchozí změny stavy hry od serveru.

Když hráč, který je na tahu, zahraje tah, pošle se na server příkaz \M{makeMove}.
Je-li tah validní, uloží se do databáze a oběma hráčům se pošle oznámení
\M{GameUpdated} o~novém stavu hry (viz sekce \ref{sec:rules-implementation}).
Toto se opakuje pro každý tah, až dokud hra neskončí.

Samotný konec hry (výhra jednoho z~hráčů) není nic víc než jen změna stavu hry
na \M{player1won} nebo \M{player2won}.

\begin{figure}[H]
    \centering
    \includegraphics[width=360px]{img/diagrams/make-move.pdf}
    \caption{Schéma komunikace zahrání dvou tahů}
    \label{fig:make-move}
\end{figure}

\subsection{Opakování hry}
Po skončení hry si uživatelé většinou chtějí zahrát další kolo
hry\footnote{nejčastější připomínka testerů}, proto se jim po skončení hry
zobrazí tlačítko \It{RENEW GAME} (obnovit hru). To na server pošle příkaz
\M{renewGame}. Server vytvoří novou hru stejným způsobem jako se hra vytváří
standardně a pošle oběma klientům zprávu \M{RenewGame} se slugem nové hry.
Front-end přesměruje prohlížeč na \M{/\{newSlug\}} a znovu se opakuje celý
proces připojení počínající příkazem \M{connectPlayer}.

\section{Implementace pravidel}\label{sec:rules-implementation}

Vedle samotného průběhy hry je potřeba umět rozpoznat, zda hráč hraje validní
tah, zda hráč svým tahem vyhrál. Server též musí umět aktualizovat stav hrací
plochy po zahraném tahu.

\subsection{Validace tahu}
Zda je hráčův tah validní v~kódu zjišťuje funkce \M{isMoveValid}. Funkce
přijímá čtyři argumenty:
\begin{enumerate}
    \item \M{grid} (pole \M{GridFieldState}): hrací plocha reprezentovaná podle
        definice v~sekci \ref{par:game-grid}
    \item \M{buttonIndex} (číslo\footnote{celé číslo, TypeScript má však pouze
        datový typ \M{number} a narozdíl od jiných jazyků mezi celým a
        desetinným číslem nerozlišuje}): ID herního políčka, na které chce hráč
        hrát, podle definice v~sekci \ref{par:game-grid}
    \item \M{player} (text \M{player1}, nebo text \M{player2}): role hráče,
        který tah chce zahrát
    \item \M{lastPos} (číslo): ID herního políčka, na které se hrálo
v~předchozím tahu, nebo $-1$ při prvním tahu
\end{enumerate}

Funkce zkontroluje, zda \M{buttonIndex} náhodou není nevalidní hodnota, zda
hráč nehraje na políčko, na které se hrálo v~předhozím tahu, a zda nehraje na
políčko, na které už hrál buď pouze on, nebo oba hráči. Pokud se ani jedno
nestane, tah je validní.

Funkce neřeší, zda hraje hráč, který je na řadě.

\begin{lstlisting}[language=JavaScript,caption={Funkce \M{isMoveValid}},label={lst:is-move-valid}]
/** @param lastPos use -1 for initial move */
const isMoveValid = (grid: GridFieldState[], buttonIndex: number, player: 'player1' | 'player2', lastPos: number) => {
    if (buttonIndex < 0 || buttonIndex > 8)
        return false;
    if (buttonIndex === lastPos)
        return false;
    const currentValue = grid[buttonIndex];
    if (currentValue === `${player}in` || currentValue === 'both')
        return false;
    return true;
};
\end{lstlisting}

\subsection{Změna hrací plochy}
Chceme-li zahrát tah, je potřeba ho zapsat do hrací plochy, to zajišťuje funkce
\M{updateGrid}. Funkce přijímá tři argumenty:
\begin{enumerate}
    \item \M{grid} (pole \M{GridFieldState}): hrací plocha reprezentovaná podle
        definice v~sekci \ref{par:game-grid}
    \item \M{buttonIndex} (číslo): ID herního políčka, na které chce hráč hrát,
        podle definice v~sekci \ref{par:game-grid}
    \item \M{player} (text \M{player1}, nebo text \M{player2}): role hráče,
        který tah chce zahrát
\end{enumerate}

Funkce získá aktuální hrací políčko a pokud do něj zatím nikdo nehrál (stav
\M{empty}\footnote{viz \ref{fig:db-schema}}), zapíše do něj, že na něj zahrál
aktuální hráč, pokud na políčko ale už hrál protihráč, zapíše na aktuální
políčko, že na něj už hráli oba hráči (stav \M{both}).

Funkce očekává, že \M{isMoveValid} proběhlo úspěšně.

Funkce neřeší ukládání stavu do databáze, pouze měnění na stav nový.

\begin{lstlisting}[language=JavaScript,caption={Funkce \M{updateGrid}},label={lst:update-grid}]
/** assumes isMoveValid === true */
const updateGrid = (grid: GridFieldState[], buttonIndex: number, player: 'player1' | 'player2') => {
    const currentValue = grid[buttonIndex]!;
    const newValue = currentValue === 'empty' ? `${player}in` as const : 'both';
    const newGrid = [...grid];
    newGrid[buttonIndex] = newValue;
    return newGrid;
};
\end{lstlisting}

\subsection{Zjišťování výhry}
Po tom, co hráč zahraje tah, je potřeba zjistit, zda nevyhrál hru. K~tomu
slouží funkce \M{isGameWon}. Funkce přijímá jeden jediný argument, tedy hrací
plochu ve stejném formátu jako předchozí dvě funkce.

Funkce si definuje vlastní interní funkci \M{allBoth}, která přijímá tři
argumenty (čísla). Interní funkce získá stavy políček určenými svými třemi
argumenty a pro každé zkontroluje, zda je stav \M{both}.

Funkce \M{isGameWon} následně zavolá interní funkce \M{allBoth} pro všechny tři
horizontály, tři vertikály a dvě diagonály. Pokud nastane jedna z~výherních
pozic, hra je vyhraná.

Funkce neřeší, kdo hru vyhrál.

\begin{lstlisting}[language=JavaScript,caption={Funkce \M{isGameWon}},label={lst:is-game-won}]
/** call after updateGrid */
const isGameWon = (grid: GridFieldState[]) => {
    const allBoth = (a: number, b: number, c: number) =>
        grid[a] === 'both' && grid[b] === 'both' && grid[c] === 'both';

    // rows
    if (allBoth(0, 1, 2))
        return true;
    if (allBoth(3, 4, 5))
        return true;
    if (allBoth(6, 7, 8))
        return true;

    // columns
    if (allBoth(0, 3, 6))
        return true;
    if (allBoth(1, 4, 7))
        return true;
    if (allBoth(2, 5, 8))
        return true;

    // diagonals
    if (allBoth(0, 4, 8))
        return true;
    if (allBoth(2, 4, 6))
        return true;

    // :(
    return false;
};
\end{lstlisting}

\section{Uživatelské rozhranní}\label{sec:user-interface}

Design stránky by po vzoru autora měl působit hravě, neseriózně a zábavně.
Proto byly při tvorbě uživatelského rozhraní použity výrazné barvy a font Comic
Sans, který je považován za jeden z~nejvíce nevhodně používaných fontů písma
\cite{seddon2011}.

Hráči, který vytvořil hru a čeká, než se připojí jeho protihráč, se zobrazuje
QR kód s~odkazem do hry, aby ho mohl druhý hráč naskenovat svým mobilním
telefonem, a odkaz do hry, který lze zkopírovat a protihráči ho poslat online.
Jednoduchá možnost sdílení herního odkazu byl jeden z~hlavních požadavků při
testování s~uživateli během vývoje. K vytvoření QR kódu se používá knihovna
\M{qrcode.react} \cite{react-qr}.

Hrací plocha je v~prohlížeči vykreslena vektorově jako SVG pomocí několika
React komponentů. V~kódu je zapsaných devět cest, které ohraničují hrací
políčka. Jednou jsou na vrchu obrázku a se používají pro rozpoznání najetí myši
na herní políčko. Podruhé jsou na spodku obrázku a používají se pro zvýraznění
herního políčka, na které uživatel najel myší. Po najetí se zvýrazňují pouze
políčka, na která uživatel může v danou chvíli kliknout.
