\chapter{Analýza a pravidla hry}

\section{Pravidla}
Pravidla hry TickoaTTwo jako první určil \textcite{jenkins22}.

Stejně jako tradiční piškvorky, hra TickoaTTwo se hraje na hrací ploše o velikosti 3x3. Hrací plocha má dle původní definice hry specifický vzhled.

% TODO obrázek hrací plochy

% TODO je určen první hráč?

Stejně jako tradiční piškvorky, hru TickoaTTwo proti sobě hrají dva hráči,
kteří se střídají v tazích. Jeden hráč na políčka v hrací ploše kreslí svislé
čáry, druhý hráč vodorovné čáry. Na každé políčko může daný hráč hrát pouze
jednou, avšak lze hrát na políčko, na které už hrál protihráč. Nelze však hrát
na políčko, na které hrál protihráč v jeho posledním tahu.

Není dáno, který hráč začíná.

Cílem každého hráče je dokončit řadu, sloupec či diagonálu, do které už hráli
oba dva hráči. Hra tedy končí ve stavu, kdy na hrací ploše je alespoň jedna
řada, sloupec či diagonála, ve které jsou tři \enquote{téčka} (svislá čára od
jednoho hráče, vodorovná čára od druhého hráče), podle toho i stylizovaný název
hry se třemi velkými písmeny T.


% počet her: na každém políčku čtyři, to však děleno 4 protože 4 rotace
% nejkratší hra 6 tahů vyhrává druhý, u klasických 5 tahů vyhrává první
% nejrychlejší hra 15 (? pět T, 4 samostatné, patnáctý ukončí), u klasických 9
